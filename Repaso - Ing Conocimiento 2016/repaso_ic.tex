%%%%%%%%%%%%%%%%%%%%%%%%%%%%%%%%%%%%%%%%% Short Sectioned Assignment
% LaTeX Template
% Version 1.0 (5/5/12)
%
% This template has been downloaded from:
% http://www.LaTeXTemplates.com
%
% Original author:
% Frits Wenneker (http://www.howtotex.com)
%
% License:
% CC BY-NC-SA 3.0 (http://creativecommons.org/licenses/by-nc-sa/3.0/)
%
%%%%%%%%%%%%%%%%%%%%%%%%%%%%%%%%%%%%%%%%%

%----------------------------------------------------------------------------------------
%	PACKAGES AND OTHER DOCUMENT CONFIGURATIONS
%----------------------------------------------------------------------------------------

\documentclass[paper=a4, fontsize=11pt]{scrartcl} % A4 paper and 11pt font size
\usepackage[T1]{fontenc} % Use 8-bit encoding that has 256 glyphs
\usepackage[utf8]{inputenc}
\usepackage{fourier} % Use the Adobe Utopia font for the document - comment this line to return to the LaTeX default
\usepackage[spanish]{babel} % English language/hyphenation
\usepackage{amsmath,amsfonts,amsthm} % Math packages
\usepackage{lipsum} % Used for inserting dummy 'Lorem ipsum' text into the template

%\usepackage{sectsty} % Allows customizing section commands
%\allsectionsfont{\centering \normalfont\scshape} % Make all sections centered, the default font and small caps

%\usepackage{fancyhdr} % Custom headers and footers
%\pagestyle{fancyplain} % Makes all pages in the document conform to the custom headers and footers
%\fancyhead{} % No page header - if you want one, create it in the same way as the footers below
%\fancyfoot[L]{} % Empty left footer
%\fancyfoot[C]{} % Empty center footer
%\fancyfoot[R]{\thepage} % Page numbering for right footer
%\renewcommand{\headrulewidth}{0pt} % Remove header underlines
%\renewcommand{\footrulewidth}{0pt} % Remove footer underlines
%\setlength{\headheight}{13.6pt} % Customize the height of the header

%\numberwithin{equation}{section} % Number equations within sections (i.e. 1.1, 1.2, 2.1, 2.2 instead of 1, 2, 3, 4)
%\numberwithin{figure}{section} % Number figures within sections (i.e. 1.1, 1.2, 2.1, 2.2 instead of 1, 2, 3, 4)
%\numberwithin{table}{section} % Number tables within sections (i.e. 1.1, 1.2, 2.1, 2.2 instead of 1, 2, 3, 4)
\setcounter{secnumdepth}{0}% Remove section numbering
\setlength\parindent{0pt} % Removes all indentation from paragraphs - comment this line for an assignment with lots of text

%----------------------------------------------------------------------------------------
%	TITLE SECTION
%----------------------------------------------------------------------------------------

\newcommand{\horrule}[1]{\rule{\linewidth}{#1}} % Create horizontal rule command with 1 argument of height

%\title{	
%\normalfont \normalsize 
%\textsc{university, school or department name} \\ [25pt] % Your university, school and/or department name(s)
%\horrule{0.5pt} \\[0.4cm] % Thin top horizontal rule
%\huge Assignment Title \\ % The assignment title
%\horrule{2pt} \\[0.5cm] % Thick bottom horizontal rule
%}

%\author{John Smith} % Your name

%\date{\normalsize\today} % Today's date or a custom date
\DeclareMathOperator*{\argmin}{arg\!\min}
\begin{document}

%\maketitle % Print the title

%----------------------------------------------------------------------------------------
%	PROBLEM 1
%----------------------------------------------------------------------------------------

\section{Tema 1}
Las conectivas satisfacen las siguientes propiedades:
\begin{itemize}
\item Conmutatividad
\item Asociatividad
\item Idempotencia
\item Distributiva
\item Identidad
\item Involución
\item De Morgan
\end{itemize}
No satisfacen ley de contradicción ni de tercio excluso. El conjunto de las $\mathcal{P}(U)$ con el $max$ y el $min$ forma un \textbf{lattice} distributivo y complementado, pero no un álgebra de bool por lo comentado antes. También forman un lattice De Morgan.
\subsection{Automorfismo}
$\varphi :[a,b]\rightarrow [a,b]$ continua, estrictamente creciente y cumpliendo $$\varphi (a)=a,\ \varphi (b)=b$$ es automorfismo del intervalo $[a,b]\subset\mathbb{R}$

\subsection{Negacion}
$c:[0,1]\rightarrow[0,1]$ es una negación difusa $\iff c(0)=1 \land c(1)=0$ y $c(x)\leq c(y), si x\geq y$\\
$c$ estricta $\iff$ creciente estricta (continua y $c(x)<c(y), x>y\  \forall x,y \in [0,1]$)\\
$c$ involutiva $\iff c(c(x))=x,\ \forall x \in [0,1]$\\
Las negaciones involutivas son negaciones fuertes Una negación involutiva es estricta\\
$c$ fuerte $\iff\exists\varphi$ automorfismo tal que $c(x)=\varphi^{-1}(1-\varphi(x))$
\subsection{Tnorma}
$T:[0,1]^2\rightarrow[0,1]$ es t-norma si cumple:\\
$T(x,1)=x,\ \forall x\in[0,1]$ Condiciones de contorno\\ $T(x,y)\leq T(z,u)$ si $x\leq z\ \land\ y\leq u$ Monotonia\\ Además de Conmutativa y Asociativa
\medbreak $T$ arquimediana si $T(x,x)<x\ \forall x \in (0,1)$\\
$T$ continua y arquimediana es positiva $\iff$ es estricta\\
Cualquier $T$ estricta es arquimediana\\
$T$ continua y arquimediana $\iff \exists f:[0,1]\rightarrow[0,\infty]$ con $f(1)=0$ estrictamente decreciente y continua tal que: $$T(x,y)=f^{(-1)}(f(x)+f(y))$$ donde 
$$f^{(-1)}=\begin{cases} f^{-1}(x) & \text{si }x\leq f(0) \\ 0 & \text{ en otro caso}
\end{cases}$$
Ejemplos de funciones generadoras son $f(x)=-log(x)$ y $f(x)=1-x$\medbreak
Si $T$ es continua tal que $T(x,c(x))=0\ \forall x\in [0,1]$ con $c$ estricta entonces $T$ es arquimediana. 
Además $T(x,y)=\varphi^{-1}(\min(\varphi(x)+\varphi(y)-1,0))$ con $c(x) \leq \varphi^{-1}(1-\varphi(x))$\\
$T$ estricta $\iff T(x,y)=\varphi^{-1}(\varphi(x)\varphi(y)),\ x,y\in[0,1]$\\
La tnorma minimo es la mayor y la unica idempotente. Es continua pero no es arquimediana. Las tnormas de la familia del producto ($P$) son continuas, arquimedianas y estrictamente positivas. Las de la familia de Lukasiewicz ($W$) son continuas y arquimedianas, pero no estrictas o positivas.
$$W(x,y)\leq P(x,y) \leq Min(x,y)$$
\subsection{Tconorma}
$S:[0,1]^2\rightarrow[0,1]$ es t-conorma si cumple:\\
$S(x,0)=x,\ \forall x\in[0,1]$ Condiciones de contorno\\ $S(x,y)\leq S(z,u)$ si $x\leq z\ \land\ y\leq u$ Monotonia\\ Además de Conmutativa y Asociativa
\medbreak $S$ arquimediana si $S(x,x)>x\ \forall x \in (0,1)$\\
$S$ nilpotente si $\exists x,y \in (0,1)$ tal que $S(x,y)=1$\\
$S$ continua y arquimediana $\iff \exists g:[0,1]\rightarrow[0,\infty]$ con $g(0)=0$ estrictamente creciente y continua tal que: $$S(x,y)=g^{(-1)}(g(x)+g(y))$$ donde 
$$g^{(-1)}=\begin{cases} g^{-1}(x) & \text{si }x\leq g(1) \\ 1 & \text{ en otro caso}
\end{cases}$$
Ejemplos de funciones generadoras son $g(x)=-log(1-x)$ y $g(x)=x$\medbreak
$S$ cumple $S(x,c(x))=1\ \forall x\in [0,1]$ con $c$ estricta $\iff \exists \varphi$ automrfismo del intervalo unidad tal que 
$S(x,y)=\varphi^{-1}(\max(\varphi(x)+\varphi(y),1))$ y $c(x)\geq \varphi^{-1}(1-\varphi(x))$\medbreak
Una t-conorma continua $S$ es estricta $\iff \exists \varphi$ tal que $$S(x,y)=\varphi^{-1}(\varphi(x)+\varphi(y)-\varphi(x)\varphi(y))$$
\bigbreak
\subsection{Dualidad}
$(T,S,c)$ es un triple De Morgan $\iff c(S(x,y))=T(c(x),c(y))$
\section{Tema 2}
\subsection{Composicion}
Sea $R$ una relacion en $U \times V$. Sea $S$ en $V \times W$.$R \circ S:$
$$\mu_{R \circ S}(u,w)= \sup_{v \in V}[\mu_R(u,v)\ast\mu_S(v,w)]$$
Donde $\sup$ es una t-conorma y $\ast$ es una tnorma.
\subsection{Implicacion}
$I:[0,1]^2\rightarrow[0,1]$ es implicacion si cumple $I(0,x)=I(x,1)=I(1,1)=1\ \forall x\in [0,1]$ y $I(1,0)=0$. Además:\\
$x \leq z \Rightarrow I(x,y) \geq I(z,y) \forall y \in [0,1]$\\
$y \leq t \Rightarrow I(x,y) \leq I(x,t) \forall x \in [0,1]$\\
Medida de cuanto de cierto es el consecuente respecto al antecedente. Otras propiedades:\\
$I(x,0) =c(x)$ es una negacion fuerte\\
$I(x,y) =I(c(y),c(x))$\\
$I(x,y) \geq y$
$I(c(x),x)=x$
\subsubsection{S-implications}
$I(x,y)=S(c(x),y)$
\subsubsection{QL-implications}
No suelen cumplir $x \leq z \Rightarrow I(x,y) \geq I(z,y) \forall y \in [0,1]$\\
$I(x,y)=S(c(x),T(x,y))$
\subsubsection{R-implications}
$I(x,y)=\sup {z|T(x,y) \leq y})$
\subsection{GMP}
Tenemos dos premisas:\\
Si $x$ es $A$ entonces $y$ es $B$\\
$x$ es $A'$\\
Consecuente: $y$ es $B'$
\subsection{GMT}
Tenemos dos premisas:\\
Si $x$ es $A$ entonces $y$ es $B$\\
$u$ es $B'$\\
Consecuente: $x$ es $A'$
\section{Tema 3}
Gradiente.
\section{Tema 4}
\subsection{Funcion penalty}
$P:[a,b]^{n+1}\rightarrow\mathbb{R}^+=[0,\infty]$ Satisface:\\
$P(\bar{x},y) \geq 0,\ \forall \bar{x}\in [a,b]^n,y\in [a,b]$\\
$P(\bar{x},y) = 0$ si $x_i = y\ \forall i=1,...,n$\\
$P(\bar{x},y$ es quasiconvexa en $y$ para cualquier $\bar{x}$:
$$P(\bar x,\lambda\cdot y_1+(1-\lambda)\cdot y_2) \leq \max(P(\bar{x},y_1),P(\bar{x},y_2))$$
\subsection{Disimilitud}
$d_R:[0,1]^2\rightarrow[0,1]$ es una función de disimilitud restringida si:
$d_R(x,y)=d_R(y,x)\ \forall x,y \in [0,1]$||
$d_R(x,y) = 1\iff x=0,y=1 \lor x=1,y=0$; es decir, $\{x,y\} = \{0,1\}$\\
$d_R(x,y) = 0\iff x=y$\\
Para cada $x,y,z\in [0,1]$, si $x \leq y \leq z$, entonces $d_R(x,y) \leq d_R(x,z)$ y $d_R(y,z) \leq d_R(x,z)$.
\subsection{Distancia}
\section{Tema 5}
\section{Tema 6}
\subsection{Agregacion}
$M:[0,1]^n\rightarrow[0,1]$\\
$M(0,...,0) = 0 \land M(1,...,1)=1$\\
Monotona no decreciente en cada componente
\subsection{REF}
$REF:[0,1]^2\rightarrow[0,1]$\\
Conmutativa\\4$=1\iff x=y$\\$ =0 \iff x=1\land y=0$ ó $x=0\land y=1$\\$REF(x,y)=REF(c(x),c(y))\ \forall x,y\in [0,1]$ si $c$ es negacion fuerte\\$\forall x,y,z\in[0,1]$ si $x\leq y \leq \Rightarrow REF(x,y)\geq REF(x,z) \land REF(y,z)\geq REF(x,z)$
\subsection{Normal $E_N$}
\section{Tema 7}
\subsection{Fukami, Baldwin y Pilsworth. Axiomas}
\subsection{Overlap}
$G:[0,1]^2\rightarrow[0,1]$ es una funcion de overlap si cumple:\\
Conmutativa\\$=0\iff xy=0$\\$=1\iff xy=1$\\Creciente y continua\\Si es asociativa es una t-norma\\
Para hablar de indice de overlap se sustituye por conjuntos
\subsection{Interpolación y regla composicional}
\paragraph{Regla composicional} En la regla composicional se considera $$B'=A'\circ R \equiv \mu_{B'}(y)=\underset{x \in X}{\max}T\{\mu_{A'}(x),\mu_R(x,y)\}$$ Donde $R$ la construimos mediante la impliacion $I(\mu_A,\mu_B)$
\paragraph{Interpolacion} Primero se calcula el grado de similitud entre $A$ y $A'$ con $c(A,A')$\\Con ello se contruye C\\Se calcula $B'=\min (B,C)$
\end{document}